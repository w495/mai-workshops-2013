\documentclass[a4paper, 12pt]{article} % лист а4, 12 кегль, тип документа — статья
\usepackage[utf8]{inputenc} % кодировка вводимого здесь текста
\usepackage[english, russian]{babel} % подключение словарей с переносами англ и рус языков
\usepackage{amssymb, latexsym, amsmath} % различные пакеты для работы с мат. символами
\usepackage{indentfirst} % каждый обзац начинать с красной строки
\usepackage{graphicx} % использование графики
\linespread{1} % межстрочный интервал

\begin{document}
\thispagestyle{empty}
\tableofcontents
\renewcommand{\contentsname}{Оглавление}

\newpage

\section{Первая часть}
Содержание первой части. Она содержит подчасти. \cite{knut}

\subsection{Первая подчасть}

\begin{equation}\label{gas_dyn}
m_0 \dfrac{p_0^*q(\lambda_0)F_0}{\sqrt{T_0^*}} = 
m_\text{с}\dfrac{p_\text{с}^*q(\lambda_\text{с})
F_\text{с}}{\sqrt{T_\text{с}^*}}, \quad\text{кг/c}.
\end{equation}

\begin{equation}\label{another_equation}
F_\text{с}{\sqrt{T_\text{с}^*}}, \quad\text{кг/c}.
\end{equation}

Содержание первой подчасти

\subsection{Вторая подчасть}

Содержание второй подчасти (ссылка на формулу \ref{another_equation})

\section{Вторая часть}

И т. д.

\newpage

\section*{Список литературы}
\begin{enumerate}
  \bibitem{knut} «Искусство программирования», том I, Дональд Кнут.
\end{enumerate}



\end{document}


